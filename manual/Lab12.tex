\documentclass{article}
\usepackage{graphicx} % new way of doing eps files
\usepackage{listings} % nice code layout
\usepackage[usenames]{color} % color
\usepackage{float}
\definecolor{listinggray}{gray}{0.9}
\definecolor{graphgray}{gray}{0.7}
\definecolor{ans}{rgb}{1,0,0}
\definecolor{blue}{rgb}{0,0,1}
% \Verilog{title}{label}{file}
\newcommand{\Verilog}[3]{
  \lstset{language=Verilog}
  \lstset{backgroundcolor=\color{listinggray},rulecolor=\color{blue}}
  \lstset{linewidth=\textwidth}
  \lstset{commentstyle=\textit, stringstyle=\upshape,showspaces=false}
  \lstset{frame=tb}
  \lstinputlisting[caption={#1},label={#2}]{#3}
}


\author{Jiasen Zhou, Jon Johnston}
\title{Lab 12: Pipeline Fetch and Decode}

\begin{document}
\maketitle

\section{Executive Summary}
Fill.

\section{Test Report}
To verify operation of these modules, this lab required one test bench. 
\begin{enumerate}
	\item Pipeline Test Bench
\end{enumerate}

\subsection{Pipeline Test Bench}
The Pipeline Test Bench contains:
\begin{enumerate}
	\item Inputs
	\begin{enumerate}
		\item Fill.
	\end{enumerate}	
	\item Outputs
	\begin{enumerate}	
		\item Fill.
	\end{enumerate}
\end{enumerate} 

\pagebreak

\begin{figure}
	\begin{center}
		\caption{Timing Diagram of the Pipeline test.}\label{fig:ert_pipelinetest}
		\includegraphics[width=1.0\textwidth]{../images/Lab12_pipelined_datapath_simulation.PNG}
	\end{center}
\end{figure}


\section{Code Appendix}
% The code appendix should include the test bench code and module code for this lab.
\Verilog{Verilog code for testing the pipelined datapath}{code:pipelinetest}{../code/2_decode/pipeline_datapath.v}

\end{document} 