\documentclass{article}
\usepackage{graphicx} % new way of doing eps files
\usepackage{listings} % nice code layout
\usepackage[usenames]{color} % color
\usepackage{float}
\definecolor{listinggray}{gray}{0.9}
\definecolor{graphgray}{gray}{0.7}
\definecolor{ans}{rgb}{1,0,0}
\definecolor{blue}{rgb}{0,0,1}
% \Verilog{title}{label}{file}
\newcommand{\Verilog}[3]{
  \lstset{language=Verilog}
  \lstset{backgroundcolor=\color{listinggray},rulecolor=\color{blue}}
  \lstset{linewidth=\textwidth}
  \lstset{commentstyle=\textit, stringstyle=\upshape,showspaces=false}
  \lstset{frame=tb}
  \lstinputlisting[caption={#1},label={#2}]{#3}
}


\author{Jiasen Zhou, Jon Johnston}
\title{Lab 9: Integrating Fetch and Decode}

\begin{document}
\maketitle

\section{Executive Summary}
The purpose of this lab is to design and simulate the datapath with the iExecute module integrated into it. Datapath.v now contains the fetch, decode, and execute stages of the datapath. With those modules, the datapath can retrieve instructions, interpret what needs to be done, and pass the necessary information to the ALU. Some signals are declared in the testbench to ensure operation since there are some stages that remain unintegrated. After comparing the Expected Results Table with each module's simulation, the lab was successful.

\section{Test Report}
To verify operation of these module, this lab requires one test bench. 
\begin{enumerate}
	\item Datapath Test Bench
\end{enumerate}



\pagebreak

\begin{figure}[H]
	\begin{center}
		\caption{Expected Results of the datapath test.}\label{fig:ert_datapathtest}
		\includegraphics[width=1.0\textwidth]{../images/DatapathExpected9.png}
	\end{center}
\end{figure}

\begin{figure}[H]
	\begin{center}
		\caption{Timing diagram for the datapath test.}\label{fig:datapathtest}
		\includegraphics[width=1.0\textwidth]{../images/DatapathSimulation.png}
	\end{center}
\end{figure}


\section{Code Appendix}
\Verilog{Verilog code for testing the datapath.}{code:regtest}{../code/2_decode/datapath.v}
\end{document} 